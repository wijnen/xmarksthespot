\documentclass{article}
\usepackage{listings}
\begin{document}

\section{Introduction}
This document aims to define all technical things about Wherigo. It is a work
in progress. If you have any suggestions for changes or additions, please send
them to me at wijnen@debian.org.

This document only specifies what the Cartridge writer can see. An Engine MAY
do extra processing and give extra feedback to the Player, but it MUST NOT do
this in a way that can be detected by the Cartridge. Any requirements and
prohibitions in this document are meant in this context.

This document intends to define all requirements and features for Cartridges.
These requirements MUST be respected by Authors (including Builders) and
Engines.


\section{History of Wherigo}
Originally, Wherigo was released as a closed source Engine with a closed source
Builder. There was no documentation about the internal format.

This document intends to change that. For the most part, it tries to describe
the original format as reverse engineered from the available programs. However,
there are a few changes, in cases where there is consensus that the original
format was a design mistake.


\section{Definitions}
Throughout this document, the following definitions are used:

\begin{enumerate}
	\item Player: the person playing the Wherigo game.
	\item Author: the person designing the Cartridge.
	\item Engine: the program used to play the game. This program does not contain information about the game itself; it needs a Cartridge to be used.
	\item Editor: the program used to create a Cartridge. This is usually a program designed for creating Wherigo cartridges, but can also be a simple text editor.
	\item Builder: an Editor with features intended to make Cartridge creation easier.
	\item Compiler: the program used to convert Cartridge source into  a file intended for playing.
	\item Cartridge: a game definition, which is processed by an Engine to present a game to the Player.  Cartridges can be in source or compiled form.
	\item Game Object: The ZCartridge object which holds information about the cartridge that is being played.
	\item Event: a function in the lua code which is called by the Engine in response to an event, such as the Player pressing a button. If the function is not defined, triggering the Event is not an error.
	\item Web Site: a central place to download Cartridges and mark them as completed.
	\item MAY: an option.
	\item MUST: a requirement.
	\item MUST NOT: a prohibition.
	\item SHOULD: a requirement in most cases, but there may be circumstances in which it is better to follow a different course.
	\item SHOULD NOT: a prohibition in most cases, but there may be circumstances in which it is better to follow a different course.
	\item Authoring Time: when the cartridge is written by the Author.
	\item Compile Time: when a cartridge is compiled by a Compiler.
	\item Boot Time: When the cartridge is started, until the UI is presented to the Player.
	\item Run Time: the main phase of the game.
	\item Table: a lua table.
\end{enumerate}

\section{Cartridge startup}
When playing a Wherigo game, the Player starts the Engine and loads a
Cartridge. The Engine MUST then perform at least the following actions, in this
order:

\begin{enumerate}
\item Create a clean lua 5.1 environment.
\item Remove \verb-debug-, \verb-package.loadlib- and \verb-io-, and everything
	from \verb-os-, except \verb-os.clock-, \verb-os.date-,
	\verb-os.difftime-, \verb-os.setlocale- and \verb-os.time-.
\item Add the ``Wherigo'' module, described below, to the environment.
\item Create Wherigo.Player, described below. Wherigo.Player.Cartridge MUST be set to nil.
\item Run the lua code from \verb-_cartridge.lua-. This returns the Game Object. This code is running during Boot Time.
\item Wait for a GPS fix and set Player.ObjectLocation to its coordinates.
\item Trigger the OnStart Event. This and everything after it is running during Run Time.
\end{enumerate}

When a saved game is restored, the same procedure must be used, except that
OnStart is not triggered.  Instead, the Engine MUST first restore all saved
variables, and then trigger OnRestore.

\section{The Lua operating environment.}

The Cartridge code runs in a clean lua 5.1 environment (with ``dangerous''
functions removed). Everything that is specific to Wherigo is in the Wherigo
module. Below is a description of all its contents. Engines MAY include more
members inside the Wherigo module for internal use. Anything that is not
defined in this document MUST NOT be used by Authors or Builders. This includes
creating new objects. Engines MUST NOT create or modify any variable outside of
the Wherigo module. In other words: the Wherigo module is owned by the Engine,
the rest is owned by the Author.

All constructors (functions which are used to create a new object) for objects
implementing the ZObject interface can be called in two ways: either with the
Game Object as the only argument, or with a Table as the only argument. The
keys in the Table are initial values for the attributes of the object that is
constructed.  It MUST have a key named \verb-cartridge-, which has the Game
Object as its value.

Unless otherwise noted, all functions are methods. This includes event
functions, which need to be defined by the Cartridge.  This means they need to
get the object they are called on as a first argument. The normal way to do
this in lua is by calling it as \verb-object:function()- instead of
\verb-object.function()-.

\newcommand\codedef[2]{\par\noindent\begin{tabular}{|l|l|}\hline\parbox{.4\textwidth}{#1}&\parbox{.5\textwidth}{#2}\\\hline\end{tabular}\par}
\newcommand\funcdef[4]{
	\par\vskip.5\baselineskip
	\par\noindent\textbf{Function:} #1 $\rightarrow$ #2\par
	\noindent\textbf{Arguments:} \parbox{.8\textwidth}{#3}\par
	\noindent\textbf{Description:} \parbox{.8\textwidth}{#4}\par
	\vskip.5\baselineskip
}
\newcommand\attrdef[3]{
	\par\vskip.5\baselineskip
	\par\noindent\textbf{Attribute:} #1 (#2)\par
	\noindent\textbf{Description:} \parbox{.8\textwidth}{#3}\par
	\vskip.5\baselineskip
}
\subsection{Constants}
\codedef{INVALID\_ZONEPOINT}{An object, used instead of a ZonePoint, to indicate a ``no coordinate''-value.}
\codedef{DETAILSCREEN\\INVENTORYSCREEN\\ITEMSCREEN\\LOCATIONSCREEN\\MAINSCREEN\\TASKSCREEN}{These constants are used with ShowScreen to focus the Player's attention to a certain part of the Engine's interface.}
\codedef{LOGDEBUG\\LOGCARTRIDGE\\LOGINFO\\LOGWARNING\\LOGERROR}{These constants are used with LogMessage, to set the type of message to log.}

\subsection{Player object}
The Player object is a ZItem which is tied to the Player. Its position is
automatically set to the device's GPS coordinates, and when the Player moves,
certain events may be triggered: OnEnter, OnLeave, OnProximity, OnDistant. The
Player object is also handled specially by some functions. This is described in
each function where this is the case.  Because the Player object is created
before the Game Object, its \verb-Cartridge- attribute is not set to the Game
Object, but to \verb-nil-.

Player.Name is set to the Player name, as defined by the Cartridge metadata.
Player.CompletionCode is set to a special code from the metadata, which can be
used to mark the Cartridge as ``Completed'' on the Web Site.

Player.PositionAccuracy is the accuracy of the GPS signal, in meters.

Player.ObjectLocation is set to \verb-INVALID_ZONEPOINT- during Boot Time.
Any code which depends on the Player location MUST NOT be run at Boot Time.
At Run Time, the coordinates MUST always be valid.

Player.InsideOfZones is a Table of Zones that the Player is inside of.

\subsection{Interfaces}
Most objects which are used by Wherigo provide an interface, depending on their
function.  Many objects provide the ZObject interface in addition to their own
more specialized interface.  Objects can be created by calling the interface's
constructor function.  This will return a new object.

All constructors start with a capital letter. In these descriptions, member
functions are shown with a hypothetical object with the interface name, but in
all lowercase letters. Argument values in square brackets are optional. Their
default values are given in the explanation.

All constructors for objects which provide the ZObject interface take either a
ZCartridge as their first argument, or a Table with a Cartridge key in it and
optionally more keys. This is not explained every time. The cartridge must be
the Game Object.

Interfaces which have an other interface name in parentheses after their own
name provide this interface in addition to the described attributes and
methods.  Documentation for those implicit members is not repeated.

\subsubsection{ZObject}
Interface for all objects. This should never be constructed directly, but only
as part of an interface which provides this interface.
\funcdef{zobject:Contains (object)}{boolean}{object: ZObject which is tested for being inside this one.}{Return true if zobject is the Container of object, or if zobject:Contains (object.Container). As a special case, if zobject is a Zone and object is Player, return IsPointInZone (Player.ObjectLocation, zobject)}
\funcdef{ZObject.made (object)}{boolean}{object: the object to be tested}{Tests whether an object is made by this constructor. It will also return true if the tested object is made by a constructor which provides the calling interface. This is not a member, it does not take a self object as first argument, and must not be called with `:'.}
\funcdef{zobject:MoveTo (container)}{nil}{container: new owner of zobject, or nil}{Remove zobject from its current container, and optionally into a new one. All Container and Inventory properties are changed accordingly.}
\attrdef{Active}{Boolean}{Non-active objects are ignored for all events, and are not shown in any UI elements.}
\attrdef{Cartridge}{ZCartridge}{The Game Object.}
\attrdef{Commands}{Table of ZCommand}{Commands which may be available with this object. Only valid for ZItem and Zone.}
\attrdef{Container}{ZObject}{ZObject that contains this object. Always a Zone, ZItem, or nil. This attribute is read-only; it is updated automatically when using MoveTo.}
\attrdef{CurrentBearing}{number}{Direction from the Player to this object. 0 if the Player contains this object.}
\attrdef{CurrentDistance}{Distance}{Distance from the Player to this object. 0 if the Player contains this object.}
\attrdef{Description}{String}{Description of this ZObject. For objects which can appear in the UI, this is shown as their details.}
\attrdef{Icon}{ZMedia}{Image which represents this ZObject. This can be used as an icon when the object is shown in a list.}
\attrdef{Inventory}{Table of ZObject}{A list of all objects which have this ZObject set as their Container. This attribute is read-only; it is updated automatically when using MoveTo. For Zones, the Player object is automatically added or removed when the player enters or leaves the zone, respectively. This happens before OnEnter or OnExit are triggered.}
\attrdef{Media}{ZMedia}{Image which represents this ZObject. This can be used when a larger image is required, usually for the object's details.}
\attrdef{Name}{String}{Name of this object. Used when presenting this object in the UI.}
\attrdef{On\textit{CommandName}}{Function}{These event callbacks are available for ZItem and Zone objects, once for each ZCommand that is available for it. \textit{CommandName} is the key of the command in the Commands Table. It is triggered by selecting the command. See ZCommand for details.}
\attrdef{Visible}{Boolean}{This ZObject is not shown in the UI unless this is set. It does respond normally to events though (if not also inactive).}

\subsubsection{ZCartridge (ZObject)}
The interface for the Game Object. There is always exactly one object
constructed with this interface. It is initialized with the metadata from the
Cartridge, and holds some tables which can be used to reach objects in the
game.
\funcdef{ZCartridge ()}{ZCartridge}{-}{Constructor. Unlike other objects providing the ZObject interface, this constructor does not need to receive the Game Object (for obvious reasons).}
\funcdef{zcartridge:GetAllOfType (type)}{Table of ZObjects}{String: type name to search for}{Return all items in zcartridge.AllZObjects of the requested type. Possible types are ZCartridge, ZItem, Zone, ZMedia, ZTask, ZTimer, and ZInput.}
\funcdef{zcartridge:RequestSync ()}{nil}{}{Save the current state of the Game. More information on saving is below.}
\attrdef{AllZObjects}{Table of ZObject}{List of all ZObjects in this Cartridge (including the Game Object itself). This attribut is read-only. It is automatically updated when ZObjects are created or destroyed.}
\attrdef{Activity}{String}{Activity from the metadata.}
\attrdef{Author}{String}{Version from the metadata.}
\attrdef{BuilderVersion}{String}{BuilderVersion from the metadata.}
\attrdef{Company}{String}{Company from the metadata.}
\attrdef{CreateDate}{String}{CreateDate from the metadata.}
\attrdef{Copyright}{String}{Copyright claim from the metadata.}
\attrdef{Icon}{ZMedia}{Icon from the metadata.}
\attrdef{LastPlayedDate}{String}{LastPlayedDate from the metadata.}
\attrdef{License}{String}{License from the metadata. Only the short part; the Long Value is not included.}
\attrdef{Media}{ZMedia}{Media from the metadata.}
\attrdef{OnEnd}{Function}{Event, called when the game ends.}
\attrdef{OnRestore}{Function}{Event, called immediately after loading a saved game. See more on saving below.}
\attrdef{OnStart}{Function}{Event, called when the game is started.}
\attrdef{OnSync}{Function}{Event, called just before saving a game. See more on saving below.}
\attrdef{PublishDate}{String}{PublishDate from the metadata.}
\attrdef{StartingLocation}{ZonePoint}{StartingLocation from the metadata.}
\attrdef{StartingLocationDescription}{String}{StartingLocationDescription from the metadata.}
\attrdef{TargetDevice}{ZMedia}{TargetDevice from the metadata.}
\attrdef{TargetDeviceVersion}{String}{TargetDeviceVersion from the metadata.}
\attrdef{UpdateDate}{String}{UpdateDate from the metadata.}
\attrdef{Version}{String}{Version from the metadata.}
\attrdef{ZVariables}{Table of strings or numbers}{See information on saving below.}

\subsubsection{ZItem (ZObject)}
An entity in the game, which can be interacted with. These can often be picked up, or talked to.
\funcdef{ZItem (Cartridge)}{ZItem}{-}{Constructor.}
\attrdef{ObjectLocation}{ZonePoint}{Current location of this ZObject.}

\subsubsection{ZonePoint}
Position on the globe.
\funcdef{ZonePoint (latitude, longitude, altitude)}{ZonePoint}{Coordinates.}{Constructor.}
\attrdef{altitude}{number}{Altitude.}
\attrdef{latitude}{number}{Latitude.}
\attrdef{longitude}{number}{Longitude.}

\subsubsection{Zone (ZObject)}
A region on the globe. The region is defined by its perimiter and a point
inside it. Every closed curve divides a sphere into two parts. One part is
inside and the other is outside. Normally, the inside part is a small region
and the outside part is most of the sphere, but that isn't a requirement. The
OriginalPoint defines which of the parts is inside and which is the outside.

As a special case, zones with no surface are allowed, including single points.
The OriginalPoint of such zones MUST be on the perimiter.  For all other zones,
the OriginalPoint MUST NOT be on the perimiter.

If the Player's coordinate changes in a big step, for whatever reason, all
callbacks that should have been called if it had changed in small steps must be
called. For example, if a Zone is constructed with State NotInRange and
positive ProximityRange and DistanceRange, and the Player is inside it,
OnDistant, OnProximity and OnEnter must all be called (in that order)
immediately.

\funcdef{Zone (Cartridge)}{Zone}{-}{Constructor.}
\attrdef{OnDistant}{Function}{Event, triggered when the Player enters the \textit{Distant} ring of a Zone. Note that this can be entered from Proximity and from NotInRange, and if Proximity is set to a nonpositive value, also from Inside.}
\attrdef{OnEnter}{Function}{Event, triggered when the Player enters the Zone.}
\attrdef{OnExit}{Function}{Event, triggered when the Player leaves the Zone.}
\attrdef{OnProximity}{Function}{Event, triggered when the Player enters the \textit{Proximity} ring of a Zone. Note that this can be entered from Inside and from Distant, and if DistanceRange is set to a value which is not larger than ProximityRange, from NotInRange.}
\attrdef{OriginalPoint}{ZonePoint}{The point which is supposed to be the origin of the Zone; it MUST NOT be on the perimiter, except for zero-surface zones, where it MUST be on the perimiter. It is by definition inside the Zone.}
\attrdef{Points}{Table of ZonePoint}{Perimiter of the zone.  The perimiter goes with line segments from each point to the next.  After the last point, a line segment back to the start is added to close the perimiter.}
\attrdef{ShowObjects}{String}{One of \textit{OnEnter}, \textit{OnProximity}, or \textit{Always}.  Defines when ZObjects which are inside the Zone are visible. When the Player is inside the Zone, they are always visible. When you are in the proximity range, they are only visible if this is set to Always or OnProximity. Otherwise they are only visible if this is set to Always.}
\attrdef{State}{String}{State of the Player with respect to this Zone. One of
	\textit{Inside}, \textit{Proximity}, \textit{Distant}, \textit{NotInRange}. The Player is inside the zone when Player.ObjectLocation is within the perimiter, or closer to it than the accuracy of the GPS receiver. Otherwise, if the distance to the perimiter minus the GPS accuracy is smaller than ProximityRange, the State is Proximity. If it is larger than ProximityRange, but smaller than DistanceRange, the State is Distant. Otherwise it is NotInRange. The Engine may change the meaning of ``the accuracy of the GPS receiver'' to avoid Players from jumping in and out of a zone only due to GPS jitter.  If not specified in the constructor, the initial value of State is NotInRange. If this is not the actual state at that time, callbacks are immediately called as if the Player has moved from NotInRange to their current location.}

\subsubsection{ZTask (ZObject)}
A task that the player can, and often should, do. There are three possible states: incomplete, done (correctly completed) and failed (complete, but incorrect).
\funcdef{ZTask (Cartridge)}{ZTask}{}{Constructor.}
\attrdef{Complete}{Boolean}{If true, the ZTask is considered completed. The UI MAY show this to the Player.}
\attrdef{Correct}{Boolean}{If Complete if false, this is ignored. Otherwise, it defines if the task was correctly completed.}

\subsubsection{ZTimer (ZObject)}
A timer which can be used to schedule one-time or periodic (interval)
callbacks. For interval timers, the Engine SHOULD call the OnTick
function at constant intervals. However, there is no guarantee that this is
always possible.
\funcdef{ZTimer (Cartridge)}{ZTimer}{}{Constructor.}
\funcdef{ztimer:Start ()}{nil}{}{Starts the timer. If it was already running, do nothing. If ztimer.Remaining is negative, it is set to ztimer.Duration. If ztimer.Remaining is 0, a Tick event is immediately triggered. To avoid dead-locks, a Cartridge MUST NOT set Duration of an interval timer to 0.}
\funcdef{ztimer:Stop ()}{nil}{}{Stops the timer. If it was not running, do nothing.}
\funcdef{ztimer:Tick ()}{nil}{}{This function is called when the timer expires. It can be called from the lua code to fake an expired timer event. Countdown timers are stopped when this is called. Interval timers are restarted.}
\attrdef{Duration}{Number}{Time in seconds. This is used when ztimer is started with a negative Remaining value. In particular, it is used as the restart value for Interval ZTimers.}
\attrdef{OnStart}{Function}{Event which is triggered when Start is called. It is not triggered when an Interval ZTimer restarts.}
\attrdef{OnStop}{Function}{Event which is triggered when Stop is called. It is not triggered when a ZTimer expires.}
\attrdef{OnTick}{Function}{Event which is triggered when the timer expires. For Interval timers, this is called after restarting the timer, so calling Stop from this event will work as expected.}
\attrdef{Remaining}{Number}{Time in seconds before it expires. If the timer is running, this value will automatically count down. This is a read-only value.}
\attrdef{Type}{String}{Must be \textit{Countdown} or \textit{Interval}. Defines whether ztimer stops or restarts when it expires.  If not specified, this defaults to Countdown.}

\subsubsection{ZCommand}
Helper object for commands on ZItems and Zones. When these are displayed, it may
be possible to run commands with them. There are two ways that this is
possible: with a different object, or on its own.

ZCommands work by being included in the Commands attribute of a Zone or ZItem.
The object that includes it is called the Owner. For commands that work with
objects, the object it works with is called the Target.

There are several options, each of which may have a different visualization.
Engines can choose how to do this visualization. The list below only gives
examples.

\begin{enumerate}
	\item Command works on its own. A single button is displayed.
	\item Command works with objects, but no candidates are available. A line of text explaining this is displayed. A candidate object is available if both its Visible and Active attributes are true.
	\item Command works with objects, and one or more objects are available. A group of buttons is displayed; one for each object.
\end{enumerate}

When a command button is clicked by the Player, a callback is triggered on the Owner ZObject.

\funcdef{ZCommand (Cartridge)}{ZCommand}{}{Constructor.}
\attrdef{CmdWith}{Boolean}{If true, the command needs an object to work on.}
\attrdef{EmptyTargetListText}{String}{The text that is displayed for commands which need an object, when none is available.}
\attrdef{Enabled}{Boolean}{If set the false, the command cannot be run. Normally, it will not be shown to the Player at all.}
\attrdef{MakeReciprocal}{Boolean}{Only used if CmdWith is true and WorksWithAll is false. If true (the default), the command can also be selected when the Target ZObject is shown.}
\attrdef{Text}{String}{The text that is shown to identify the command. For example, \textit{Talk}. This must be a short text. It is often identical to the name used to register the command with an object.}
\attrdef{WorksWithAll}{Boolean}{Only used if CmdWith is true. If this is true, every ZItem, except the object that this command is for, will automatically be in the list of available targets.}
\attrdef{WorksWithList}{Table of ZObjects}{Only used if CmdWith is true and WorksWithAll is false. This is a list of candidate ZItems and Zones for using the command with.}

\subsubsection{ZMedia (ZObject)}
Interface for objects which reference external data: a sound or an image.

The metadata describes which files or URLs are used to get the external media,
including the compiler directives for selecting and modifying it for use on the
target device. All ZMedia objects are created as part of the ZCartridge
construction. Lua code MUST NOT create ZMedia objects itself.

ZMedia are used from lua code, by linking them to messages (for images) or
using them with PlayAudio (for sounds). They are opaque objects; they have no
public methods or attributes.

\subsubsection{ZInput (ZObject)}
Helper object to request input from the Player. There are three types of input:
button, multiple choice and text. A button input will present a message to the
Player and wait for a button to be pressed. A multiple choice input is similar,
but there can be multiple buttons to choose from. A text input requests a line
of text from the Player.

When requesting the input, the Engine is normally using the image in
zinput.Media as an illustration, if there is one.
\funcdef{ZInput (Cartridge)}{ZInput}{}{Constructor.}
\attrdef{OnGetInput}{Function}{Event which is triggered when a multiple choice or text input is answered. It is passed a string argument: the text on the button that was pressed for a multiple choice input; the entered text for a text input. If the input is cancelled, this event is triggered with nil as its argument.}
\attrdef{Text}{String}{Text which is displayed to request the input.}

\subsection{Functions}
\subsubsection{User interface}
These functions may not be called during cartridge initialization.

There is always at most one graphical function waiting for Player input. If a
second one is called before the previous one was answered, it will replace it.
The previous message's callback function will be called with a nil argument to
signify that this happened.

\funcdef{Alert ()}{nil}{}{Give a signal to the Player, such as a beep or a flash.}
\funcdef{Dialog (table)}{nil}{table: a list of messages.}{Show a list of messages to the player, in sequence. The elements of the Table are the same as the argument of the MessageBox function, except that they cannot have a Callback or Buttons (the Engine would normally give them an implicit Ok button to show the next message).}
\funcdef{DriveTo (zonepoint)}{nil}{target ZonePoint}{Leave the Engine interface temporarily, and let the device navigate the Player to zonepoint.}
\funcdef{GetInput (zinput)}{nil}{The ZInput to request.}{Request a previously defined input from the Player.}
\funcdef{LogMessage (message)}{nil}{Message to log}{Write a message to a log file. The Engine MAY display this file to the Player as well. The log format is described below.}
\funcdef{MessageBox (table)}{nil}{table: the message to display.}{The Table, like a ZInput, contains a Text to display, and an optional Media containing an image. In addition, it contains Buttons, which is a Table of strings, which are printed on buttons, and a Callback function, which is called when a button is pressed. Callback is passed a string 'Button1' if the first button is pressed, 'Button2' for the second, etc.}
\funcdef{PlayAudio (sound)}{nil}{The ZMedia to play.}{Play sound (which must be a sound media file). Any previously playing file MAY be stopped.}
\funcdef{SaveClose ()}{nil}{}{Save the game and quit the Engine.}
\funcdef{ShowScreen (which, [item])}{nil}{which: One of the SCREEN-constants.\\item: a ZObject}{Try to focus the Player's attention on the given part of the interface. If which is DETAILSCREEN, item MUST be used to specify of which object the details should be shown. For other values of which, item MUST be ignored and should be omitted.}
\funcdef{ShowStatusText (text)}{nil}{message string}{Let the Player know about something in a non-invasive way. Normally, this happens by showing the text in the status bar of the Engine.}
\funcdef{StopSound ()}{nil}{}{Abort all currently playing sounds.}

\subsubsection{Spherical arithmetics}
These functions SHOULD use spherical arithmetics to perform their task. At the very least, they MUST work properly for zones crossing any coordinate boundary.
\funcdef{IsPointInZone (point, zone)}{Boolean}{point: The ZonePoint to test.\\zone: The Zone te test.}{Return true if point and zone.OriginalPoint are in the same segment of the sphere.}
\funcdef{TranslatePoint (point, distance, bearing)}{ZonePoint}{point: point to translate.\\distance, bearing: vector to translate over.}{Return the ZonePoint where you end up when moving from point over vector.}
\funcdef{VertorToPoint (point1, point2)}{Distance, number}{point1, point2: ZonePoints}{Return the vector from point1 to point2.}
\funcdef{VectorToSegment (point, s1, s2)}{Distance, number}{point: the ZonePoint to test.\\s1, s2: the ZonePoint endpoints of the segment to test. The segment is the shortest line over the sphere from s1 to s2. s1 and s2 MUST NOT be exactly opposite each other.}{Determine the closest point on the segment from point; return the vector from point to this point.}
\funcdef{VectorToZone (point, zone)}{Distance, number}{point: the ZonePoint to test.\\zone: the Zone to test.}{Return the minimum of the results of VectorToSegment for each of the segments of the perimiter of the Zone.}

\subsubsection{Other}
\funcdef{NoCaseEquals (s1, s2)}{Boolean}{s1, s2: strings to compare.}{Return the case-insensitive comparison of s1 and s2.}
\funcdef{ToUserUnits (distance)}{String}{Distance to convert, in meters.}{Return distance in a suitable format for displaying to the user.}

\section{File types}
\subsection{Zipped source: gwz}
A zip archive containing a single directory, which must have the same name as
the gwz file itself, minus its extension. This directory contains the gwi
metadata file, \verb-_cartridge.gwi-, and \verb-_cartridge.lua-. In addition,
if it contains sounds they must be in a subdirectory named Sound, and if it
contains images they must be in a sudirectory named Image.  If it contains
translations they must be in a subdirectory named Translation, in the form of
countrycode.po, for example \verb-nl_BE.po- for Flemmish (Belgian Dutch).

\subsection{Compiled Cartridge: gwc}
%Format defined at \verb-http://code.google.com/p/wherugo/wiki/GWC-. Additions will be required for supporting the new metadata format.
The gwc format is identical to the gwz format, with the following differences.
The file \verb-_cartridge.lua- is compiled.  The file \verb-_cartridge.gwi- is
adjusted to contain only what the target Engine can support, and all media
files are explicitly defined in it.  The Translation directory contains mo
files instead of po files.  All indented lines, including implicitly indented
empty lines, in the gwi file are indented with only a single space.

\subsection{Log: gwl}
One line per message. Each line contains a timestamp, GPS information, and a string as passed to LogMessage. It has fields separated by \verb-|-. The fields are:
\begin{enumerate}
	\item Timestamp; 14 digits without separators: Year (4), month (2), day (2), hour (2), minutes (2), seconds (2).
	\item Latitude, in degrees. Positive means north.
	\item Longitude, in degrees. Positive means east.
	\item Altitude in meters.
	\item GPS precision in meters.
	\item Message, which can be any string not containing a newline. Note that it can contain \verb-|-, which MUST NOT be treated as a field separator.
\end{enumerate}

\subsection{Metadata: gwi}
There are several media files, and there is metadata about them in this file.

The cartridge and media metadata in the gwi-file is given in RFC822-like form;
the same that is used in e-mail headers. Every field consists of one line which
matches the regular expression:

\begin{center}\verb-'^(\w+):\s*([^;]*?)(?:;\s*(\w+)\s*(?:=\s*([^;]+))?\s*)*$'-\end{center}

The subexpressions are named:
\begin{enumerate}
	\item Key
	\item Value
	\item Attribute
	\item Attribute Value
\end{enumerate}

For clarity, an example Cartridge follows below.

This first line is sometimes followed by one or more indented lines. These
lines together, including the newlines, but excluding the indentation, are
called the Long Value. Empty lines are considered indented empty lines, even if
they don't have any indentation.  Such lines (with or without indentation) are
stripped from the end of a long value, but preserved in other places.

Now follows a list of keys, with their details. Every key MUST occur exactly
once, unless otherwise specified. Attributes not defined here MUST NOT be
used. Values for fields which do not allow Attributes may contain any
characters, including semicolons (this exception is not shown in the regular
expression above).

\newcommand\defkey[5]{\par\noindent\begin{tabular}{|l|l|}\hline Key&#1\\Value&\parbox{.75\textwidth}{#2}\\Long Value&\parbox{.75\textwidth}{#3}\\Attributes&\parbox{.75\textwidth}{#4}\\Comments&\parbox{.75\textwidth}{#5}\\\hline\end{tabular}\par}
\defkey{Name}{The cartridge's long name. The filename, which is the short name, SHOULD be an abbreviation of this long name.}{The cartridge's description.}{-}{}
\defkey{Version}{The cartridge's version.}{-}{-}{The version SHOULD be a period-separated string. Versions released later SHOULD have a higher version than earlier versions of the same Cartridge. Comparing versions is done by splitting the version on periods and comparing each group numerically if possible, and alphabetically otherwise. The first group which differs defines which version is higher. For example, \textit{1.2.8} is a lower version than \textit{1.2.10} and both are lower than \textit{1.2.unreleased}}
\defkey{Author}{The Author.}{-}{-}{}
\defkey{E-mail}{The e-mail address for sending feedback.}{-}{-}{}
\defkey{Copyright}{The copyright claims for this cartridge, or empty if all copyright claims are in the long value.}{Continuation of the copyright claims, if required.}{-}{This should list the people or organizations who hold the copyright on this cartridge.}
\defkey{License}{The name of the license under which this cartridge is distributed.}{Full license text, or a link to it, for the cartridge source. Note that this information is not included in the compiled Cartridge, so it MUST be extracted from the source and passed on to the location where the compiled Cartridge can be retrieved.}{-}{The license is a statement from the copyright holder which allows the Web Site (and possibly the Player) to distribute the cartridge.}
\defkey{Company}{The name of the company or other organization that the author works for.}{-}{-}{This key is optional.  If used, the company should normally also be set as copyright holder.}
\defkey{Activity}{MUST be one of a defined set of values. Currently allowed values are TourGuide, Puzzle, Fiction, and Geocache.}{-}{-}{}
\defkey{StartingLocation}{Three floating point numbers, setting the starting location of the Cartridge. The first two numbers define the latitude and longitude, both in degrees. The third number defines the altitude, in meters.}{Description of the starting location.}{-}{}
\defkey{BuilderVersion}{Version of the Builder.}{-}{-}{This key is optional. A Builder MUST include this key. The version has the same requirements as Cartridge Version, plus the requirement that the first group of the version MUST be the Builder's name.}
\defkey{Media}{Name of the title image.}{-}{-}{This key is optional. If given, it must be the name of a (possibly implictly) defined Image.}
\defkey{Icon}{Name of the icon image.}{-}{-}{This key is optional. If given, it must be the name of a (possibly implictly) defined Image.}
\defkey{CreateDate}{Date that the cartridge was originally created.}{-}{-}{}
\defkey{UpdateDate}{Date that the cartridge was last updated.}{-}{-}{}
\defkey{Image}{Name of the image.}{Description of the image.}{AltText = text to use when the image cannot be displayed.}{See discussion about media below.}
\defkey{Sound}{Name of the sound.}{Description of the sound.}{-}{See discussion about media below.}
\defkey{File}{Filename of a media object.}{-}{Directives for this file; format to be determined.}{See discussion about media below.}
\defkey{URL}{URL where a media file can be retrieved.}{-}{Directives for this file; format to be determined.}{See discussion about media below.}

\vskip\baselineskip

Media objects are created by the Engine when the ZCartridge is created. They
are reachable from lua as members of the Game Object's \textit{Sound} and
\textit{Image} members.

The definition of media files (Image, or Sound) in the gwi-file is done
by first defining their name, using the key of the appropriate type.
Immediately after this, zero or more File or URL fields must be defined to
specify options which provide this media object. File and URL fields MUST NOT
be used in any other place.

Filenames specified by File keys are relative to the Image or Sound
subdirectory of the Cartridge source, for Image and Sound media files
respectively. If they use subdirectories, they MUST use the forward slash
``\verb-/-'' as a directory separator.

Media objects or files which do not need to specify anything, do not need to be
defined in the gwi-file. A media file which is in the base Image or Sound
directory of a Cartridge and which is not defined in the gwi-file MUST be
treated the same as a media file for which an Image, or Sound field (depending
on the file location) is defined as the filename of the file, minus the
extension, with a corresponding File with the filename as a value. If there are
several files which would define the same media object (they differ only in
their file extension), they MUST be treated as files all providing the same
object. (But if they are different types, one Sound and one Image, they MUST be
providing separate objects.)

Media files which are placed in subdirectories of Image or Sound MUST be
referenced using an explicit File field.

If a media object is defined, but no sources, then any files of a fitting type
with the object's name as their basename are used as candidates to provide it.

If a media object is defined and no sources are present, it MUST be treated the
same way as if there are sources, but none of them can be used. This is not an
error: for images, the Alt text MUST be displayed instead of the image; for
sounds, requests to play them SHOULD play an Engine-defined default sound.

\subsection{Example cartridge}
\subsubsection{Lua file}
\noindent\begin{lstlisting}
require "Wherigo"

cartridge = Wherigo.ZCartridge ()

function cartridge:OnStart ()
	Wherigo.PlayAudio (cartridge.Sound.bell)
	Wherigo.Dialog {{Text = "Look at me!",
		Media = cartridge.Image.dude},}
end

return cartridge
\end{lstlisting}
\subsubsection{gwi file}
\noindent\begin{lstlisting}
Name: Example Cartridge
	This is an example cartridge header,
	showing how it is built up.
	This description can be shown when
	selecting which cartridge to play.
Version: 1.0
Author: Bas Wijnen
E-mail: <wijnen@debian.org>
Copyright: Copyright 2013 Bas Wijnen
License: AGPL-3+
	You may copy, modify and redistribute
	this cartridge under the terms of the
	GNU Affero General Public License,
	version 3 or later (at your option).
	See http://www.gnu.org/licenses/agpl-3.0.html
	for details.
Activity: TourGuide
StartingLocation: 53.0 6.0 0
	Go there and see farmland!
\end{lstlisting}

In the file, bell.wav and dude.jpg are also packaged. They do not need to be
mentioned in the gwi-file.

\subsection{Saved game: ?}
The format for this file still has to be decided, if it is going to be
standardized at all.
\end{document}
